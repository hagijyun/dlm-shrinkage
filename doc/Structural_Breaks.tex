\documentclass{article}

\usepackage{amsmath}
\usepackage{amsfonts}

\author{Jeffrey B. Arnold}
\title{Time Varying Parameter Estimation Robust to Structural Breaks}
\date{March 4, 2013}

\begin{document}



Consider a univariate dynamic linear model,
$$
\begin{aligned}
y_t &= \theta_t + \nu_t \\
\theta_t &= \theta_{t-1} + \omega_t
\end{aligned}
$$
where $E(\nu_t) = 0$ and $E(\omega_t) = 0$. Since I am primarily
concerned about structural breaks, for simplicity, I will assume
$\nu_t \sim N(0, \sigma_\nu)$.%
\footnote{However, the use of fat-tailed distributions could also be
  applied to identify outliers.}

The nature of the change in $\theta$ is determined by the distribution on $\nu_t$.  
In a constant model, $\theta$ is not time-varying and thus, $\omega_{t} = \delta_{0}$, where $\delta_{0}$ is the degenerate distribution at zero.
In a normal dynamic linear model, $\omega_{t}$ are distributed normal, $\omega_{t} \sim N(0, \sigmua_{omega})$.
Since the normal distribution does not have thick tails, it smooths the evolution of $\theta$ over time. 
This means that 


\end{document}

%%% Local Variables: 
%%% mode: latex
%%% TeX-master: t
%%% End: 
