\documentclass{article}

\usepackage{amsmath}
\usepackage{amsfonts}
\usepackage[margin=1in]{geometry}
\usepackage{fancyvrb}
% \usepackage{color}
\usepackage[latin1]{inputenc}
\usepackage[style=authoryear,backend=bibtex8]{biblatex}
\usepackage{graphicx}
\usepackage{subcaption}
\addbibresource{default}
%\addbibresource{local}
\usepackage{setspace}
\doublespace

\usepackage{todonotes}

\author{Jeffrey B. Arnold}
% Seeing a Shrink about Structural Breaks
\title{Sparse State Disturbance Dynamic Linear Models for Structural Breaks}


% Used to typeset distributions
\newcommand{\dist}[1]{\mathcal{#1}}
\newcommand{\paren}[1]{\ensuremath{\left(#1\right)}}
\newcommand{\dnorm}[1]{\ensuremath{\dist{N}\paren{#1}}}
\newcommand{\dmvnorm}[2]{\ensuremath{\dist{N}_{#2}\paren{#1}}}
\newcommand{\dt}[2]{\ensuremath{\dist{T}_{#1}\paren{#2}}}
\newcommand{\dcauchy}[1]{\ensuremath{\dist{C}\paren{#1}}}
\newcommand{\dhalfcauchy}[1]{\ensuremath{\dist{C}^{+}\paren{#1}}}
\newcommand{\dbeta}[1]{\ensuremath{\dist{B}\paren{#1}}}
\newcommand{\dinvbeta}[1]{\ensuremath{\dist{IB}\paren{#1}}}
\newcommand{\dgamma}[1]{\ensuremath{\dist{G}\paren{#1}}}
\newcommand{\dinvgamma}[1]{\ensuremath{\dist{IG}\paren{#1}}}
\newcommand{\dwishart}[1]{\ensuremath{\dist{W}\paren{#1}}}
\newcommand{\dinvwishart}[1]{\ensuremath{\dist{IW}\paren{#1}}}
\newcommand{\dunif}[1]{\ensuremath{\dist{U}\paren{#1}}}
\newcommand{\dexp}[1]{\ensuremath{\dist{E}\paren{#1}}}

\newcommand{\RLang}{\textsf{R}}
\newcommand{\Stan}{Stan}
\newcommand{\R}{\ensuremath{\mathbb{R}}} %real

\DeclareMathOperator{\E}{E}
\DeclareMathOperator{\Var}{Var}
\DeclareMathOperator{\Cov}{Cov}
\DeclareMathOperator{\diag}{diag}
\newcommand{\tran}{^\top}


\makeatletter
\def\PY@reset{\let\PY@it=\relax \let\PY@bf=\relax%
    \let\PY@ul=\relax \let\PY@tc=\relax%
    \let\PY@bc=\relax \let\PY@ff=\relax}
\def\PY@tok#1{\csname PY@tok@#1\endcsname}
\def\PY@toks#1+{\ifx\relax#1\empty\else%
    \PY@tok{#1}\expandafter\PY@toks\fi}
\def\PY@do#1{\PY@bc{\PY@tc{\PY@ul{%
    \PY@it{\PY@bf{\PY@ff{#1}}}}}}}
\def\PY#1#2{\PY@reset\PY@toks#1+\relax+\PY@do{#2}}

\expandafter\def\csname PY@tok@gu\endcsname{\let\PY@bf=\textbf}
\expandafter\def\csname PY@tok@gs\endcsname{\let\PY@bf=\textbf}
\expandafter\def\csname PY@tok@cm\endcsname{\let\PY@it=\textit}
\expandafter\def\csname PY@tok@gp\endcsname{\let\PY@bf=\textbf}
\expandafter\def\csname PY@tok@ge\endcsname{\let\PY@it=\textit}
\expandafter\def\csname PY@tok@cs\endcsname{\let\PY@it=\textit}
\expandafter\def\csname PY@tok@gh\endcsname{\let\PY@bf=\textbf}
\expandafter\def\csname PY@tok@ni\endcsname{\let\PY@bf=\textbf}
\expandafter\def\csname PY@tok@nn\endcsname{\let\PY@bf=\textbf}
\expandafter\def\csname PY@tok@s2\endcsname{\let\PY@it=\textit}
\expandafter\def\csname PY@tok@s1\endcsname{\let\PY@it=\textit}
\expandafter\def\csname PY@tok@nc\endcsname{\let\PY@bf=\textbf}
\expandafter\def\csname PY@tok@ne\endcsname{\let\PY@bf=\textbf}
\expandafter\def\csname PY@tok@si\endcsname{\let\PY@bf=\textbf\let\PY@it=\textit}
\expandafter\def\csname PY@tok@nt\endcsname{\let\PY@bf=\textbf}
\expandafter\def\csname PY@tok@ow\endcsname{\let\PY@bf=\textbf}
\expandafter\def\csname PY@tok@c1\endcsname{\let\PY@it=\textit}
\expandafter\def\csname PY@tok@kc\endcsname{\let\PY@bf=\textbf}
\expandafter\def\csname PY@tok@c\endcsname{\let\PY@it=\textit}
\expandafter\def\csname PY@tok@sx\endcsname{\let\PY@it=\textit}
\expandafter\def\csname PY@tok@err\endcsname{\def\PY@bc##1{\setlength{\fboxsep}{0pt}\fcolorbox[rgb]{1.00,0.00,0.00}{1,1,1}{\strut ##1}}}
\expandafter\def\csname PY@tok@kd\endcsname{\let\PY@bf=\textbf}
\expandafter\def\csname PY@tok@ss\endcsname{\let\PY@it=\textit}
\expandafter\def\csname PY@tok@sr\endcsname{\let\PY@it=\textit}
\expandafter\def\csname PY@tok@k\endcsname{\let\PY@bf=\textbf}
\expandafter\def\csname PY@tok@kn\endcsname{\let\PY@bf=\textbf}
\expandafter\def\csname PY@tok@kr\endcsname{\let\PY@bf=\textbf}
\expandafter\def\csname PY@tok@s\endcsname{\let\PY@it=\textit}
\expandafter\def\csname PY@tok@sh\endcsname{\let\PY@it=\textit}
\expandafter\def\csname PY@tok@sc\endcsname{\let\PY@it=\textit}
\expandafter\def\csname PY@tok@sb\endcsname{\let\PY@it=\textit}
\expandafter\def\csname PY@tok@se\endcsname{\let\PY@bf=\textbf\let\PY@it=\textit}
\expandafter\def\csname PY@tok@sd\endcsname{\let\PY@it=\textit}

\def\PYZbs{\char`\\}
\def\PYZus{\char`\_}
\def\PYZob{\char`\{}
\def\PYZcb{\char`\}}
\def\PYZca{\char`\^}
\def\PYZam{\char`\&}
\def\PYZlt{\char`\<}
\def\PYZgt{\char`\>}
\def\PYZsh{\char`\#}
\def\PYZpc{\char`\%}
\def\PYZdl{\char`\$}
\def\PYZhy{\char`\-}
\def\PYZsq{\char`\'}
\def\PYZdq{\char`\"}
\def\PYZti{\char`\~}
% for compatibility with earlier versions
\def\PYZat{@}
\def\PYZlb{[}
\def\PYZrb{]}
\makeatother


\begin{document}

\maketitle{}


\section{Introduction}
\label{sec:introduction}

Political and social processes are rarely, if ever, constant over time.
Thus political scientists often have a need to estimate time-varying parameters (TVP).
There exist two broad approaches to estimating time-varying parameters: structural break approaches, including dummy variables and change point models, and smoothing approaches, including dynamic linear models and smoothing splines.
These two approaches are viewed as distinct and estimated using different methods, forcing the researcher to choose between the models.
Since in many processes the changes in the process are characterized by many periods of stability and a few periods of possibly rapid and large, change \parencite{RatkovicEng2010},
which 
structural break (change point) models are often used \parencites{CalderiaZorn1998}{Spirling2007a}{Spirling2007b}{Park2010}{Park2011}.%
\footnote{\textcite{RatkovicEng2010} is the notable exception in that their method combines both smoothly varying sections with structural breaks.}
However, many in general the structural break methods become more difficult to formulate and estimate as the number of breaks goes from known to estimated.

This paper presents a simple and flexible method to estimate time-varying parameters that may be subject to structural breaks.
Time-varying parameters with possible structural breaks can be estimated within a continuous state-space (dynamic linear model) model by placing a shrinkage prior on the distribution of the state disturbances.
The intuition behind this can be illustrated with a simple model.
Suppose there is a vector of observed data, $y_{1}, \dots, y_{n}$, drawn from a normal distribution with a time varying mean, $\alpha_{1:n}$. 
This can be represented within a state space model as follows,
\begin{equation}
  \label{eq:4}
  \begin{aligned}[t]
    y_{t} &= \alpha_{t} + \varepsilon_{t} & \varepsilon_{t} \sim N(0, \sigma^{2}) \\
    \alpha_{t + 1} &= \alpha_{t} + \eta_{t}
  \end{aligned}
\end{equation}
The difference between "structural breaks" and "smoothing" data-generating processes and estimation techniques is whether the $\eta$ vector is assumed (estimated) to be sparse (most $\eta_{t} = 0$) or dense (most $\eta_{t} \neq 0$).
The commonly estimated local level dynamic linear model specifies a common normal distribution for $\eta$.
While this can estimate change over time, it cannot capture sparse $\eta$, as it will tend to over-smooth the breaks and under-smooth the periods of stability.
However, estimating sparse parameter vectors is a general problem that has received considerable attention lately in large-p, small-n problems (include citations).
This paper applies some of those advances to estimating time-varying parameters, such as \eqref{eq:4}.
Instead of using a prior normal distribution on $\eta$, a shrinkage prior is used instead.
While there are a large number of Bayesian shrinkage priors proposed, this paper will use the Horseshoe Prior distribution introduced in \textcites{CarvalhoPolsonScott2009}{CarvalhoPolsonScott2010}.

Using a sparse disturbance representation of TVP models has multiple favorable characteristics.
\begin{enumerate}
\item This method does not require specifying the number of structural breaks ex ante.
Structural breaks can be detected from $\eta$ using several rules.
The sparsity of $\eta$ will determine the number of structural breaks, and this sparsity can be 
Not only do the number of structural breaks need not be specified beforehand, this method will work reasonable well even if the underlying data-generating process has the parameter changing in each period.
\item This method is flexible.
Dynamic linear models incorporate a wide variety of models, including ARIMA and structural time-series, cubic splines, and regressions with time-varying coefficients.
Any model in which the parameter of interest can be expressed as a latent state in a dynamic linear model can be altered to assign a shrinkage prior to the state disturbance in order to make it robust to or to detect structural breaks for that parameter.
\item The method allows for easy estimation of structural breaks in multiple parameters which can be either independent or correlated.
\item Outliers can be included in the same model by applying shrinkage priors to the observation disturbances instead of the state disturbances in the dynamic linear model.
\item This method is efficient in both programmer and computational time, while still retaining the flexibility to estimate a wide variety of models.
Since many most shrinkage priors, including the Horseshoe Prior used in the paper, are scale-mixture of normal distributions, this method can take advantage of the computationally efficient methods of mode finding and sampling from dynamic linear models, such as the Kalman filter and Forward-Filter Backwards-Sample.
This paper shows how a combination of \Stan, a general purpose Bayesian software program, and \RLang can be used to easily estimate and sample from the posterior of dynamic linear models.
\end{enumerate}


\section{Dynamic Linear Models}
\label{sec:dynam-line-models}

A Dynamic Linear Model (DLM), also called a Linear Gaussian state space model, for a $n$-dimensional observation sequence $y_{1}, \dots, y_{n}$ is defined by the following set of equations.%
\footnote{This paper follows the notation used in \textcite{DurbinKoopman2001}. See \textcite{PetrisPetroneEtAl2009} for a concordance with the notation used in \textcite{WestHarrison1997}.}
For $t = 1:n$,
\begin{align}
  \label{eq:8}
  \underset{p \times 1}{y_t} &= \underset{p \times m}{Z_{t}} \, \underset{m \times 1}{\alpha_t} + \underset{p \times 1}{\varepsilon_t} & \varepsilon_{t} &\sim \dmvnorm{0, H_{t}}{p} \\
  \lable{eq:14}
  \underset{m \times 1}{\alpha_{t+1}} &= \underset{m \times m}{T_{t}} \, \underset{m \times 1}{\alpha_{t}} + \underset{m \times r}{R_{t}}  \underset{r \times 1}{\eta_{t}} & \eta_{t} &\sim \dmvnorm{0, Q_{t}}{r} \\
  \label{eq:2}
  \alpha_{1} & \sim \dmvnorm{a_{1}, P_{1}}{m}
\end{align}

Equation \eqref{eq:8} is the \textit{observation equation} which relates the \textit{observation vector} $y_{t}$ to the \textit{state vector} $\alpha_{t}$.
Equation \eqref{eq:14} is the \textit{state equation} which describes the Markovian evolution of the state vector.
Equation \eqref{eq:2} is the \textit{initial state equation} which is a prior distribution for the initial state $\alpha_{1}$.
The vectors $\varepsilon_{t}$ and $\eta_{t}$ are referred to as the \textit{observation} and \textit{state disturbances}, respectively.
The matrices $Z_{t}$, $H_{t}$, $T_{t}$, $R_{t}$, and $Q_{t}$ are referred to as the \textit{system matrices}.
Let $\mathcal{S}_{t}$ refer to the set of system matrices.
The matrix $Z_{t}$ is the design matrix, $T_{t}$ is the transition matrix, $H_{t}$ is the observation covariance matrix, and $R_{t} Q_{t} R'_{t}$ is the state covariance matrix.
For the purposes of the estimation of $\alpha$, they are considered fixed and known, but in a larger model they can include parameters to be estimated.
%Let $p$ be the dimension of the observation vector (number of variables), $m$ be the dimension of the state vector, and $r$ be the dimension of the state disturbances.
%then the dimensions of the elements in the DLM are show in Table \ref{tab:state_space_dim}.
% \begin{table}[!]
%   \centering
% \begin{tabular}{llll}
%   Vectors & dimension & Matrices & dimension \\
% \hline
%   $y_t$     & $p, 1$ & $Z_{t}$  & $p, m$ \\
%   $\alpha_{t}$ & $m, 1$ & $H_{t}$  & $p, p$ \\
%   $\varepsilon_t$ & $p, 1$ & $T_{t}$ & $m, m$ \\
%   $\eta$ & $r, 1$ & $R_t$ & $m, r$ \\
%    &  & $Q_t$ & $r, r$ \\
%   $a_{1}$ & $m, 1$ & $P_{1}$ & $m, m$
% \end{tabular}
%   \caption{The imensions of the elements in DLM (Equations \eqref{eq:8}, \eqref{eq:14}, and \eqref{eq:2})}
%   \label{tab:state_space_dim}
% \end{table}

Many common models, including ARIMA and structural time-series, regressions with time-varying coefficients, cubic splines, and stochastic volatility models can be represented as DLMs. 
For thorough treatments of DLMs and state space models see \textcites{WestHarrison1997}{DurbinKoopman2001}{PetrisPetroneEtAl2009}{CommandeurKoopman2007}.
This flexibility is important because the methods presented below can be applied to a wide range of models.
Any model in which the parameter of interest can be represented as a state in a DLM, can be estimated using the following method to detect structural breaks.


\subsection{Structural Breaks in a State Space Model}
\label{sec:struct-breaks-state}

\subsubsection{Local Level Model}
\label{sec:local-level-model}

For simplicity of exposition, in this section I will restrict my attention to a univariate observation vector $y_{1:n}$ with a time-varying mean.
This can be represented as a dynamic linear model as follows,
\begin{equation}
  \label{eq:5}
  \begin{aligned}[t]
    y_{t} &= \alpha_{t} + \varepsilon_{t} & \varepsilon & \sim \dnorm{0, H_{t}} \\
    \alpha_{t + 1} &= \alpha_{t} + \eta_{t} & \eta & \sim \dnorm{0, Q_{t}}
  \end{aligned}
\end{equation}
Equation \eqref{eq:5} is the special case of the dynamic linear model in \eqref{eq:8} with $p = m = r = 1$, and $Z_{t} = T_{t} = 1$.
The model in Equation \eqref{eq:5} is commonly called the \textit{local level model}, or a \textit{random walk with noise} \textcite[Chapter 2][]{DurbinKoopman2001}[Chapter 2][]{WestHarrison1997}.

In equation \eqref{eq:5}, first difference in the state are equal to the value of the state disturbance, $\Delta \alpha_{t} = \alpha_{t+1} - \alpha_{t} = \eta_{t}$.
Thus, the distribution of parameter differences is the distribution of $\eta_{t}$.
In many problems in political and social science, the evolution of a parameter is expected to to be stable for long periods of time, with a few changes of possibly large magnitude, i.e. the structural breaks \parencite{Pierson2004}[57][]{RatkovicEng2010}.
Translated into the the language \eqref{eq:5}, this statement means that the researcher has a prior belief that the vector $\eta$ is or may be sparse.
I will refer to a vector as sparse if most elements are 0, and dense if most elements are not zero.
It is expected that for most periods $\eta_{t} = 0$, but there are a few periods in which $\eta_{t} \neq 0$, and, possibly even periods in which $|\eta_{t}| \gg 0$.
When cast in this way, the problem of estimating structural breaks within a dynamic linear model is essentially a problem of estimating a sparse parameter vector $\eta$.
Estimating sparse parameters, especially when the number of observations small relative to the number of observations (the large-p, small-n) is a problem that is presently receiving much attention in the areas of penalized or regularized regression, variable selection, and multiple-testing. (CITE)
Following the convention in that literature, I will refer $\eta_{t} \approx 0$ as ``noise'' and $|\eta_{t}| \gg 0$ as ``signals''.
In this application, the structural breaks are the signals, and the non-structural breaks are the noise.
In sparse estimation problems, the researcher wants to classify parameters into signals and noise, and estimate the magnitude of the signals.

In Bayesian estimation, there are two main approaches for estimating sparse parameters: discrete mixtures and shrinkage priors, which differ in the class of prior distributions placed on the parameters \textcite[73]{CarvalhoPolsonScott2009}.
The first approach uses a discrete mixture of a point mass at zero and a continuous distribution, i.e. a spike-and-slab prior (Mithcell Beauchamp, George McCulloch).
This approach includes methods such as variable selection priors and Bayesian model selection and averaging.
In the context of estimating TVP, structural breaks can be estimated by assigning each $\eta_{t}$ a discrete mixture distribution,
\begin{equation}
  \label{eq:1}
  \eta_{t} = p \delta_{0} +  (1 - p) g(\eta_{t})
\end{equation}
where $p \in [0, 1]$, $\delta_{0}$ is the degenerate distribution at 0, and $g(\eta_{t})$ is the distribution if $\eta_{t} \neq 0$.
\textcite{GiordaniKohn2008} propose using \eqref{eq:1} as a flexible model of structural breaks within a continuous state space framework.

The second approach to estimating sparse parameters is shrinkage priors (Tibishirami, Tipping)
Shrinkage priors are absolutely continuous distributions centered at zero.
Although shrinkage priors do not mix between signal and noise groups, they are able to approximate the mixture distribution if they have the following features,
\begin{itemize}
\item a large mass near zero to shrink noise observations
\item heavy tails to keep signals unshrunk.
\end{itemize}
Shrinkage priors are the Bayesian posterior estimation equivalent of penalized likelihood for posterior mode-finding.
For example, the popular Lasso/L1 regularization corresponds to \textit{maximum a posteriori} (MAP) estimation with a Laplacian (double exponential) prior on the parameters.
Other prior distributions correspond to other, sometimes implicit and lacking analytic form, likelihood penalties.

Almost all proposed and used shrinkage priors can be represented as scale mixtures of normal distributions \parencite{PolsonScott2010}.
Thus, the prior distribution on the state distribution can be represented a normal distribution centered zero, in which the variance term in which the product of a global variance component $\tau^{2}$ and a local variance component $\lambda^{2}$,
\begin{equation}
  \label{eq:3}
  \begin{aligned}[t]
    \eta_{t} &= N(0, \tau^{2} \lambda_{t}^{2}) \\
    \lambda_{t}^{2} &\sim p(\lambda_{t}^{2})
  \end{aligned}
\end{equation}
Shrinkage priors differ in the distribution of the local variance component.%
For example, commonly used shrinkage priors that are scale mixtures of normal distributions include the Student's t-distribution ($\lambda^{2}_{t} \sim \dinvgamma{a, b}$) and the double exponential distribution (Bayesian Lasso) ($\lambda^{2}_{t} \sim \dexp{2}$) \textcite[74]{CarvalhoPolsonScott2009}.
\footnote{
Note that the commonly used $\eta_{t} \sim N(0, Q)$ is trivially a scale-mixture of normal distributions ($\lambda_{t}^{2} = \delta_{1}$).
However, it does not impose sparsity as it neither has heavy tails nor does it have a large mass at zero.
}

That most shrinkage priors can be represented as scale mixture of normal distributions is essential for the computational efficiency in estimating DLMs with sparse disturbances.
This ensures that conditional on $\lambda_{t}$, the dynamic model is still a Gaussian dynamic linear model, although technically a Conditionally Gaussian Dynamic Linear Model (CGDLM) (CITE).
As discussed in more detal in section \ref{sec:implementation}, this means that the dynamic linear model component of the model can be sampled using efficient block sampling methods that depend on $\eta_{t}$ and $\varepsilon_{t}$ being distributed normal.

While there are a large number of proposed shrinkage priors \textcite{ArmaganDunsonLee2011}\textcite{BrownGriffin2010}\textcite{PolsonScott2010}, this paper will use the Horseshoe Prior distribution \parencites{CarvalhoPolsonScott2009}{CarvalhoPolsonScott2010}{PolsonScott2010}{PolsonScott2012}{DattaGhosh2012}.
The Horseshoe Prior distribution has no analytical form, but can be represented as a scale-mixture of normal distributions (as in \eqref{eq:3}) in which the local scale components $\lambda$ are distributed half-Cauchy,%
\footnote{
The half-Cauchy prior on the scale components $\lambda_{t}$ implies an inverse-Beta (Beta prime) distribution on the variance components $\lambda_{t}^{2} = \dinvbeta{\frac{1}{2}, \frac{1}{2}}$, where $\dinvbeta{x; a, b} = \frac{x^{a - 1} (1 + x)^{-a - b}}{B(a, b)}$, where $B$ is the Beta function.
The inverse beta distribution is related to the beta distribution.
If $X \sim \dbeta{a, b}$, then $\frac{1}{1 + X} \sim \dinvbeta{a, b}$.
}
\begin{equation}
  \label{eq:6}
  \lambda \sim \dhalfcauchy{0, 1}
\end{equation}
The Horseshoe prior distribution has many appealing properties as a shrinkage prior distribution, which follow from its infinitely high spike at zero, and heavy Cauchy-like tails.
This allows it to aggressively shrink noise towards zero, while not shrinking signals.%
\footnote{See \textcite{CarvalhoPolsonScott2010}{CarvalhoPolsonScott2009}{DattaGhosh2012} for formal properties of the horseshoe prior.}
Figure \ref{fig:horseshoe} compares the density function of the Horseshoe Prior to the normal, Cauchy, and Laplacian (double-exponential) distributions, both around zero and in the tails.
\begin{figure}
  \centering
  % \includegraphics{plots/fig-horseshoe1.pdf}
  % \includegraphics{plots/fig-horseshoe2.pdf}
  \caption{The density of the horseshoe prior distribution (in black) compared with the densities of the normal, Cauchy, and Laplacian distributions (in gray).}
  \label{fig:horseshoe}
\end{figure}

Using shrinkage priors to model structural breaks extends and generalizes the use of the $t$-distribution for modeling structural breaks \textcites{HarveyKoopman2000}[184][]{DurbinKoopman2001}{PetrisPetroneEtAl2009}.
The $t$-distribution is also a scale-mixture of normal distributions and has heavy tails, but does not have a substantial mass near zero, and thus will insufficiently shrink noise distributions.

To summarize, the proposed local level model that is robust to structural breaks is as follows,
\begin{equation}
  \label{eq:10}
  \begin{aligned}[t]
    y_{t} &\sim N(\alpha_{t}, H) \\
    \alpha_{t + 1} &\sim N(\alpha_{t}, \tau^{2} \lambda^{2}_{t}) \\
    \lambda^{2}_{t} & \sim \dhalfcauchy{0, 1}
  \end{aligned}
\end{equation}
This model could be completed with the non-informative priors, suggested in \textcite{CarvalhoPolsonScott2009} and {PolsonScott2010},
\begin{equation}
  \label{eq:7}
  \begin{aligned}[t]
    p(H) &= \frac{1}{H} \\
    \tau &\sim \dhalfcauchy{0, H}
  \end{aligned}
\end{equation}
$H$ is given its Jeffrey's prior, while $\tau$ is distributed Cauchy on the same scale as $H$.

\subsection{General DLMs}
\label{sec:multivariate}

The previous section developed the sparse disturbance DLM for the local level case, but the use of shrinkage priors on the state disturbances can easily be extended to the more general DLM in section \ref{sec:dynam-line-models}.
First, consider the case in which the state disturbances are uncorrelated; let $Q_{t} = \diag(q_{1}, \dots, q_{m})$.%
\footnote{
  For simplicity of exposition, I will assume that $m = r$ and $R_{t} = I_{m}$.
}
\todo{Should I keep the $R Q R'$ notation or use a single covariance matrix?}
Then, each $\eta_{t,i}$ is distributed,
\begin{equation}
  \label{eq:20}
  \eta_{t,i} \sim \dnorm{0, \lambda_{t,i}^{2} \tau_{i}}
\end{equation}
where $\lambda$ is a $n \times r$ matrix, and $\tau$ is a $r \times 1$ vector.

Next, consider the case in which the state disturbances, $\eta_{t}$ are correlated.
In that case, the state disturbances are distributed as a scale mixture of multivariate normal distributions.
The multivariate extension of equation \eqref{eq:3} is
\todo{This derivation of the scale mixture of MVN seems intuitive, but I haven't seen it done}
\begin{equation}
  \begin{aligned}[t]
    \eta_{t} &\sim \dmvnorm{0, \Lambda_{t} \Gamma \Lambda_{t}'}{m} \\
    \Lambda \Lambda' & \sim p(\Lambda \Lambda')
  \end{aligned}
\end{equation}
where $\Gamma$ is the global covariance component, and $\Lambda_{t}$ is a lower triangular matrix such that  $\Lambda_{t} \Lambda_{t}'$ is the local covariance matrix.
The state disturbances $\eta_{t}$ can be correlated either at the global or local scale.
A non-diagonal $\Lambda$ indicates that the incidence of structural breaks are correlated.

To generalize the Horseshoe Prior distribution to matrix variate random variables, I decompose $\Lambda_{t} \Lambda_{t}'$ into a standard deviation vector ($\lambda_{t}$) and a correlation matrix ($R_{t}$),%
\begin{align}
  \label{eq:16}
  \Lambda_{t} \Lambda_{t}' &= \diag(\lambda_{t}) R_{t} \diag(\lambda_{t})' \\
  \label{eq:17}
  \lambda_{t,i} &= \dhalfcauchy{0, 1} & \text{for $i \in 1:p$}
\end{align}
For all $t \in 1:n$, each element of $\lambda_{.,i}$ shares a common prior distribution, which shrinks $\lambda_{1:t,i}$ over time.
Non-zero elements in $R_{t}$ indicate that the mixing parameters are correlated, meaning that the structural breaks are correlated.

\subsection{Structural Breaks}

Given posterior estimates from a sparse disturbance dynamic linear model, there are a few methods that can be used to detect structural breaks.

The first method uses the posterior distribution of $\eta_{t}$ to detect structural breaks in the corresponding $\alpha$.
A structural break can be classified as an $\eta_{t}$ in which which the (95\%) credible interval of the posterior distribution excludes zero.
This is the Bayesian equivalent to the auxiliary residual test of \textcites{JongPenzer1998}{DurbinKoopman2001}.%
Apart from differences between credible and confidence intervals, the Bayesian method differs from the auxiliary residual test in that it marginalizes over the posterior distribution of the state matrices $S_{t}$ rather fixing them at point estimates.
Note that in some applications, especially those in which the transition matrix $T_{t}$ includes parameters, it may make more sense to use the posterior distribution of the difference in the states, $p(\alpha_{t} - \alpha_{t - 1} | y)$.

The second method uses the local variance components $\lambda_{t}$ to identify structural breaks \parencite[179-180]{PetrisPetroneEtAl2009}.
If $\lambda_{t} = 1$, then the state disturbance is distributed normal with a scale equal to the global scale $\eta_{t} \sim N(0, \tau^{2})$.
Thus, the local shrinkage parameters $\lambda$ are relative measure of how much each $\eta_{t}$ is shrunk towards zero.
Values of $\lambda_{t} > 1$ ($\lambda_{t} < 1$) indicate state distrurbances that are dispersed (shrunk) relative to the global scale.
A structural break can be classified as a $t$ in which $\E (p(\lambda_{t} | y, .)) > 1$.
Alternatively, the probability of a structural break is $\Pr(p(\lambda_{t} | y, .) > 1)$.

The examples in section \ref{sec:examples} will use both of these methods.

\subsection{Outliers}
\label{sec:outliers}

Most of the discussion on modeling structural breaks applies to modeling outliers.
Outliers can be modeled with a DLM by placing a shrinkage prior on the observation disturbances $\varepsilon$ instead of the state disturbance $\eta$.
A difference difference between observation disturbances and state disturbances, is that there is usually a higher prior belief of sparsity in state disturbances than there is in observation disturbances. 
Almost always observation disturbances are expected to be non-zero.
Outliers are not the few periods in which $\varepsilon \neq 0$, but are instead the few periods with observation disturbances of very large magnitudes, $|\varepsilon| \gg 0$.
In this case it is more important for the shrinkage prior to have heavy tails than have a spike at zero.
For that reason, the $t$-distribution, which is commonly used for outliers in Bayesian regression and in DLMs, will likely work well in most applications.

\section{Examples}
\label{sec:examples}


\subsection{Nile River Flow}
\label{sec:nile}

The Nile data is a series of readings of the annual flow volume of the Nile River at Aswan taken between 1871 and 1970.
This dataset has been analyzed and used as an example in many works on structural breaks and DLMs \parencites{Cobb1978}{Balke1993}{JongPenzer1998}{DurbinKoopman2001}{DurbinKoopman2012}.
Previous analyses show a single level shift in the data, with the shift occurring in 1899.
This level shift was due to the construction of a damn at Aswan that year.

For the Nile River model, I will compare three models. 
All of these models will be variants of the local level model,
\begin{equation}
  \label{eq:21}
  \begin{aligned}[t]
    y_{t} &\sim \dnorm{\alpha_{t}, H_{t}} \\
    \alpha_{t + 1} &\sim \alpha_{t} + \eta_{t}
  \end{aligned}
\end{equation}
The models differ in the distribution they place on $\eta_{t}$.
The first model, $M_{nile,hp}$ places a horseshoe prior distribution on the state disturbances,
\begin{equation}
  \label{eq:22}
  \begin{aligned}[t]
    \eta_{t} & \sim \dnorm{0, \lambda^{2}_{t} \tau_{t}^{2}} & \lambda_{t} & \sim \dhalfcauchy{0, 1}
  \end{aligned}
\end{equation}
The second model $M_{nile,normal}$ is the standard local level model in which the $\eta$ are drawn i.i.d. from a normal distribution as in \textcite{DurbinKoopman2001} and \textcite{petris2011state},
\begin{equation}
  \label{eq:9}
  \begin{aligned}[t]
    \eta_{t} & \sim \dnorm{0, \tau_{t}^{2}}
  \end{aligned}
\end{equation}
The third model, $M_{nile,inter}$ includes a manual intervention in 1899 ($t = 28$),
\begin{equation}
  \label{eq:12}
  \begin{aligned}[t]
    \eta_{t} & \sim \dnorm{0, \tau_{t}^{2} + \zeta^{2}} \\
    \zeta & = 
    \begin{cases}
      10^{6} & \text{if $t = 28$} \\
      0 & \text{otherwise}
    \end{cases}
  \end{aligned}
\end{equation}
By setting the variance of $\eta_{28}$ very large, the state discounts all previous observations and is able to immediately adjust to the new data.
\footnote{An alternative method for incorporating an intervention is to change the observation equation to
  \begin{equation}
    \label{eq:23}
    y_{t} \sim( \beta x_{t} + \alpha_{t}, \sigma^{2})
  \end{equation}
  where $x_{t}$ is an indicator variable such that $x_{t} = 1$ if $t \geq 28$ and 0 otherwise.
}

All models use the following priors distributions of $\tau$ and $\sigma$ (ADD CITATION AND EXPLANATION)
\begin{align}
  \label{eq:24}
  p(\sigma) &= \frac{1}{\sigma} \\
  \label{eq:25}
  \tau &\sim \dhalfcauchy{0, \sigma}
\end{align}
All the models use an informative, empirical prior for the initial state,
\begin{align}
  \label{eq:13}
  \alpha_{1} &\sim N(y_{1}, \Var y_{1})
\end{align}

% \begin{figure}[htpb]
%   \centering
%   \includegraphics{plots/fig-nile.pdf}
%   \caption{Plot of mean posterior predictive distributions ($\E p(\tilde{y}| y)$) for the normal, normal2, and horseshoe prior distribution models.}
%   \label{fig:nile}
% \end{figure}

% \begin{figure}[htpb]
%   \centering
%   \includegraphics{plots/fig-nile_innovations.pdf}
%   \caption{Plot of innovations}
%   \label{fig:nile_innovations}
% \end{figure}

% \begin{table}[htpb]
%   \centering
%   % latex table generated in R 3.0.0 by xtable 1.7-1 package
% Sun May  5 02:57:42 2013
\begin{tabular}{rrrrrr}
  \hline
 & WAIC & $p$ & $L$ & MSE & $\chi^2$ \\ 
  \hline
normal & 1259 & 15 & -614 & 11999 & 68 \\ 
  normal2 & 1259 & 6 & -623 & 15060 & 90 \\ 
  hs & 1261 & 9 & -622 & 14478 & 82 \\ 
   \hline
\end{tabular}

%   \caption{Model summary statistics of Nile models.}
%   \label{tab:nile}
% \end{table}

\subsection{George W. Bush Approval Ratings}
\label{sec:george-w.-bush}

The motivating example in \textcite{RatkovicEng2010}.

\subsection{Third Example: TBD}

\begin{itemize}
\item Turning points in greenback prices. A replication of \textcite{WillardGuinnaneEtAl1996}.
\item Median ideal point of the Senate. \parencites{RatkovicEng2010}
\item Supreme court dissents and concurrences. 1 or 2 structural breaks. Poisson data. \parencite{CalderiaZorn1998}
\item Discrete DV change-point models in \parencite{Spirling2007b}.
\item Interest Rates, Inflation, and GDP growth are common economics examples, e.g. \textcite{GiordaniKohn2008}.
\end{itemize}

\section{Implementation}
\label{sec:implementation}

MCMC sampling from a dynamic linear model is challenging due to the temporal dependence of the latent state parameters $\lambda_{t}$ \parencite{ReisSalazarGamerman2006}.
Although full conditional distributions for a Gibbs sampler can be easily specified (Carlin et al 1992), in practice sampling component-wise will almost always result in slow convergence and highly correlated posterior samples.
However, in the case of dynamic linear models there exist more efficient block sampling algorithms which sample from $\alpha_{t}$ in a single block.
These methods include the Forward-Filter Backward Smoothing algorithm of \textcite{CarterKohn1994} and \textcite{Fruehwirth-Schnatter1994}, as well as more efficient simulation smoothers of \textcite{DeJongShephard1995}, \textcite{DurbinKoopman2002}, \textcite{StricklandTurnerDenhamEtAl2009}.%
\footnote{Also see \textcite{ReisSalazarGamerman2006} for a comparison of the efficiency of various sampling methods, and \textcite{migon2005dynamic} for an overview of the various sampling methods.}
Additionally, since in the case of a Gaussian dynamic linear models the posterior $p(\alpha | y)$ is multivariate normal with a sparse, block diagonal covariance matrix, samples can be drawn directly from the posterior distribution \parencites{migon2005dynamic}{ChanJeliazkov2009}).

However, the efficient algorithms are not included in the commonly used general-purpose Bayesian software programs (BUGS, JAGS, and PyMc).
Thus, to estimate and sample from dynamic linear models in a computationally efficient manner, the researcher must write a custom MCMC sampler.
While there exists many software implementations of filters and smoothers for Dynamic Linear Models, sampling from other parameters may require additional sampling tricks due to the correlation between the states and the parameters (Weis, and citations, Chan Jeliazkov).

To estimate the models in this paper, I used a combination of Stan and R.

Stan is a general purpose Bayesian software program.
Like BUGS, it has a domain specific language that lets the user specify the statistical model without specifying the steps used to sample from the model.
Unlike BUGS, Stan is not based on Gibbs sampling, but instead uses a variant of Hamiltonian Monte Carlo (citation).
While as of the time of the writing, a distribution for dynamic linear models is not included in the software, but a method for efficiently sampling the other parameters while marginalizing over the latent states can be implemented within the Stan modeling language.
The objective in the Stan step is draw posterior samples from (any parameters in) the state matrices $S_{t}$.
Think of dynamic linear model as  $p(y | S_{t})$. 
To sample from a distribution not built into Stan, all that is required is the computation of the log likelihood of that distribution, which is added to the log-posterior.
In the case of dynamic linear models, the log-likelihood of $p(y | S_{t})$ can be efficiently calculated with the Kalman filter (citations).

Thus, to get a sample from the
\begin{enumerate}
\item Sample $k$ iterations from the posterior distribution of $p(S_{t} | y_{t})$ by using a Kalman filter implemented in Stan to calculate $\log p(y_{t} | S_{t})$.
\item For each sample from the posterior distribution of $p(S_{t} | y_{t})$, draw $\alpha | y_{t}, S_{t}$ using an efficient sampling algorithm.
This paper used the R package KFAS (cite).
\end{enumerate}

Separating the steps in this manner allows for efficient sampling of parameters in $S$ with HMC, while marginalizing over $\alpha$.
If samples from $\alpha$ are needed, then they can be sampled after, possibly in parallel.


\section{Conclusion}
\label{sec:conclusion}

This paper shows that a simple tweak to dynamic linear models allow them to estimate structural break models with a random number of breaks.
The sparse disturbance approach is both intuitive and flexible, while remaining computationally efficient.

\clearpage{}

\printbibliography{}

\end{document}

%%% Local Variables: 
%%% mode: latex
%%% TeX-master: t
%%% End: 

%  LocalWords:  Carvallho TVP RatkovicEng CalderiaZorn Spirling eq DV
%  LocalWords:  CarvalhoPolsonScott ARIMA DLM DurbinKoopman unshrunk
%  LocalWords:  PetrisPetroneEtAl WestHarrison GiordaniKohn Laplacian
%  LocalWords:  PolsonScott DLMs DattaGhosh Jeffrey's MVN JongPenzer
%  LocalWords:  Balke nile petris ReisSalazarGamerman CarterKohn PyMc
%  LocalWords:  Fruehwirth Schnatter smoothers DeJongShephard migon
%  LocalWords:  StricklandTurnerDenhamEtAl ChanJeliazkov MCMC Weis
%  LocalWords:  Jeliazkov KFAS HMC spirling bayesian
