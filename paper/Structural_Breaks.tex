\documentclass{article}

\usepackage{amsmath}
\usepackage{amsfonts}
\usepackage[margin=1in]{geometry}
\usepackage{fancyvrb}
% \usepackage{color}
\usepackage[latin1]{inputenc}
\usepackage[style=authoryear]{biblatex}
\usepackage{graphicx}
\usepackage{subcaption}
\addbibresource{local}
\usepackage{setspace}
\doublespace

\author{Jeffrey B. Arnold}
% Seeing a Shrink about Structural Breaks
\title{Scale Mixture Innovation Models as a Flexible Model of Time-Varying Parameters with Structural Breaks}

\newcommand{\paren}[1]{\ensuremath{\left(#1\right)}}
\newcommand{\dnorm}[1]{\ensuremath{\mathcal{N}\paren{#1}}}
\newcommand{\dmvnorm}[2]{\ensuremath{\mathcal{N}_{#2}\paren{#1}}}
\newcommand{\dt}[2]{\ensuremath{\mathcal{T}_{#1}\paren{#2}}}
\newcommand{\dcauchy}[1]{\ensuremath{\mathcal{C}\paren{#1}}}
\newcommand{\dhalfcauchy}[1]{\ensuremath{\mathcal{C}^{+}\paren{#1}}}
\newcommand{\dbeta}[1]{\ensuremath{\mathcal{B}\paren{#1}}}
\newcommand{\dinvbeta}[1]{\ensuremath{\mathcal{IB}\paren{#1}}}
\newcommand{\dgamma}[1]{\ensuremath{\mathcal{G}\paren{#1}}}
\newcommand{\dinvgamma}[1]{\ensuremath{\mathcal{IG}\paren{#1}}}
\newcommand{\dwishart}[1]{\ensuremath{\mathcal{W}\paren{#1}}}
\newcommand{\dinvwishart}[1]{\ensuremath{\mathcal{IW}\paren{#1}}}
\newcommand{\dunif}[1]{\ensuremath{\mathcal{U}\paren{#1}}}

\newcommand{\RLang}{\textsf{R}}
\newcommand{\R}{\ensuremath{\mathbb{R}}} %real

\DeclareMathOperator{\E}{E}
\DeclareMathOperator{\var}{Var}
\DeclareMathOperator{\cov}{Cov}


\makeatletter
\def\PY@reset{\let\PY@it=\relax \let\PY@bf=\relax%
    \let\PY@ul=\relax \let\PY@tc=\relax%
    \let\PY@bc=\relax \let\PY@ff=\relax}
\def\PY@tok#1{\csname PY@tok@#1\endcsname}
\def\PY@toks#1+{\ifx\relax#1\empty\else%
    \PY@tok{#1}\expandafter\PY@toks\fi}
\def\PY@do#1{\PY@bc{\PY@tc{\PY@ul{%
    \PY@it{\PY@bf{\PY@ff{#1}}}}}}}
\def\PY#1#2{\PY@reset\PY@toks#1+\relax+\PY@do{#2}}

\expandafter\def\csname PY@tok@gu\endcsname{\let\PY@bf=\textbf}
\expandafter\def\csname PY@tok@gs\endcsname{\let\PY@bf=\textbf}
\expandafter\def\csname PY@tok@cm\endcsname{\let\PY@it=\textit}
\expandafter\def\csname PY@tok@gp\endcsname{\let\PY@bf=\textbf}
\expandafter\def\csname PY@tok@ge\endcsname{\let\PY@it=\textit}
\expandafter\def\csname PY@tok@cs\endcsname{\let\PY@it=\textit}
\expandafter\def\csname PY@tok@gh\endcsname{\let\PY@bf=\textbf}
\expandafter\def\csname PY@tok@ni\endcsname{\let\PY@bf=\textbf}
\expandafter\def\csname PY@tok@nn\endcsname{\let\PY@bf=\textbf}
\expandafter\def\csname PY@tok@s2\endcsname{\let\PY@it=\textit}
\expandafter\def\csname PY@tok@s1\endcsname{\let\PY@it=\textit}
\expandafter\def\csname PY@tok@nc\endcsname{\let\PY@bf=\textbf}
\expandafter\def\csname PY@tok@ne\endcsname{\let\PY@bf=\textbf}
\expandafter\def\csname PY@tok@si\endcsname{\let\PY@bf=\textbf\let\PY@it=\textit}
\expandafter\def\csname PY@tok@nt\endcsname{\let\PY@bf=\textbf}
\expandafter\def\csname PY@tok@ow\endcsname{\let\PY@bf=\textbf}
\expandafter\def\csname PY@tok@c1\endcsname{\let\PY@it=\textit}
\expandafter\def\csname PY@tok@kc\endcsname{\let\PY@bf=\textbf}
\expandafter\def\csname PY@tok@c\endcsname{\let\PY@it=\textit}
\expandafter\def\csname PY@tok@sx\endcsname{\let\PY@it=\textit}
\expandafter\def\csname PY@tok@err\endcsname{\def\PY@bc##1{\setlength{\fboxsep}{0pt}\fcolorbox[rgb]{1.00,0.00,0.00}{1,1,1}{\strut ##1}}}
\expandafter\def\csname PY@tok@kd\endcsname{\let\PY@bf=\textbf}
\expandafter\def\csname PY@tok@ss\endcsname{\let\PY@it=\textit}
\expandafter\def\csname PY@tok@sr\endcsname{\let\PY@it=\textit}
\expandafter\def\csname PY@tok@k\endcsname{\let\PY@bf=\textbf}
\expandafter\def\csname PY@tok@kn\endcsname{\let\PY@bf=\textbf}
\expandafter\def\csname PY@tok@kr\endcsname{\let\PY@bf=\textbf}
\expandafter\def\csname PY@tok@s\endcsname{\let\PY@it=\textit}
\expandafter\def\csname PY@tok@sh\endcsname{\let\PY@it=\textit}
\expandafter\def\csname PY@tok@sc\endcsname{\let\PY@it=\textit}
\expandafter\def\csname PY@tok@sb\endcsname{\let\PY@it=\textit}
\expandafter\def\csname PY@tok@se\endcsname{\let\PY@bf=\textbf\let\PY@it=\textit}
\expandafter\def\csname PY@tok@sd\endcsname{\let\PY@it=\textit}

\def\PYZbs{\char`\\}
\def\PYZus{\char`\_}
\def\PYZob{\char`\{}
\def\PYZcb{\char`\}}
\def\PYZca{\char`\^}
\def\PYZam{\char`\&}
\def\PYZlt{\char`\<}
\def\PYZgt{\char`\>}
\def\PYZsh{\char`\#}
\def\PYZpc{\char`\%}
\def\PYZdl{\char`\$}
\def\PYZhy{\char`\-}
\def\PYZsq{\char`\'}
\def\PYZdq{\char`\"}
\def\PYZti{\char`\~}
% for compatibility with earlier versions
\def\PYZat{@}
\def\PYZlb{[}
\def\PYZrb{]}
\makeatother


\begin{document}

\maketitle{}

\begin{abstract}
  In estimating problems with time-varying parameters, researchers often have to choose between methods that smooth the change over time and methods that model the change as discrete breaks.
  This paper proposes using dynamic linear models with scale-mixture of gaussians a model time-varying processes that can account for either or both smoothly time-varying processes and processes with large discrete jumps.
  The problem of estimating time-varying parameters is a special case of the ``large-p'' problem, and this paper applies recent advances in that literature to the time-varying parameter problem.
  This provides a robust and flexible method. 
\end{abstract}

\section{Introduction}
\label{sec:introduction}

Political processes vary over time, and the variation is often marked by periods of stability interrupted by large changes \parencite{RatkovicEng2010}.
Thus, there is a need for methods to estimate time-varying parameters and those methods must be able to handle sparse changes in the parameters.
Two approaches to estimating time-varying parameters are structural-break/change-point/regime-switching models and smoothing models.


This paper shows how dynamic linear models can easily be extended to be robust to and identify structural breaks.

The incorporation of structural breaks and the general problem of estimating time-varying parameters is related to and a special case of the large-p, small-n problem.
Suppose there are $T$ time periods, and the researcher would like to estimate a parameter $\theta$ that can take a different values in each period, $\theta_{t}$ for $t \in 1:T$.
The problem is that there are often only $T$ data points, meaning that some additional structure must be applied to estimate 
The time-varying parameter problem is different than many other large-p, small-n problems due to the ordering of the parameters.

First, both smoothing and structural breaks can be represented within the continuous state space form. 

Modeling time-varying parameters with structural breaks with scale-mixtures of normals is an approach that has favorable properties both in terms of its flexibility, ease of implementation, and computational efficiency.
\begin{itemize}
\item Unlike many existing discrete state-space approaches, the number of change points (or equivalently the sparsity in the changes in the parameters) does not need to be specified \textit{ex ante}, but can be estimated.
\item This method is flexible.
  This is a special case of a conditional gaussian dynamic linear models (CGDLM) \parencites{WestHarrison1997}{DurbinKoopman2001}{CommandeurKoopman2007}{ShumwayStoffer2010}.
  DLMs are a class of models that include regression, stochastic volatility, and ARIMA models.
  The structural break method presented here can easily be adapted to model changes in multiple parameters, seasonal effects, slopes, and variance.
\item It is convenient to independently model changes in multiple parameters, a situation which would cause an explosion in the number of states in discrete state space framework (Hidden Markov).
\item Since the horseshoe prior distribution is hierarchical distribution formed from normal and Cauchy distributions, this model can easily be specified in general purpose Bayesian software (BUGS, Stan, etc.), even though that is not the most efficient method of estimating it.
\end{itemize}

This paper makes several contributions to the discussion of time-varying parameters within the political science literature.
First, it places structural break and smoothing methods within a common modeling framework.
This is useful in its own right; increases the understanding of structural breaks by focusing on the distribution of the changes of the state, and suggests fututure routes forward for building time-varying parameter models.
Second, it links time-varying parameter models to the large-p, small-n literature (regularized / penalized regression in the likelihood literature, and shrinkage and variable selection in the Bayesian literature). 

\section{Dynamic Linear Models}

A Dynamic Linear Model (DLM) is defined by the following set of equations,
\begin{align}
  \label{eq:8}
  Y_t &= F_{t} \theta_t + \nu_t & \nu_{t} &\sim \dmvnorm{0, V_{t}}{m} \\
  \label{eq:14}
  \theta_t &= G_{t} \theta_{t-1} + \omega_{t} & \omega_{t} &\sim \dmvnorm{0, W_{t}}{p} \\
  \label{eq:2}
  \theta_{0} & \sim \dmvnorm{m_{0}, C_{0}}{p}
\end{align}  
where $t = 1:T$, 
$Y_{t}$ and $\nu_{t}$ are length $m$ vectors,
$\theta_{t}$, $m_{0}$, and $\omega_{t}$ are length $p$ vectors,
$F_{t}$ is a $m \times p$ matrix, 
$V_{t}$ is an $m \times m$ matrix,
and $G_{t}$, $C_{0}$ and $W_{t}$ are $p \times p$ matrices.
Equation \eqref{eq:8} is called the \textit{observation} equation, 
equation \eqref{eq:14} is called the \textit{system} equation,
and \eqref{eq:4} is the prior distribution on the initial state.
I will refer to the $\nu_{t}$ as observation \textit{errors}, and the $\omega_{t}$ as \textit{innovations}.

DLMs nest a large number of common models, including ARIMA, stochastic volatility, (time-varying parameter) regressions,
and cubic splines \parencites{WestHarrison1997}{DurbinKoopman2001}\parencite{PetrisPetroneEtAl2009}{CommandeurKoopman2007}.%
\footnote{See \textcite{CommandeurKoopmanOoms2011} for a review of statistical software to estimate state space models.}
This flexibility is important because the methods presented below can be applied to a wide range of models with little modification.

Perhaps the most simple non-trivial DLM is the the univariate local level model,
\begin{align}
  \label{eq:15}
  y_t &= \theta_t + \nu_t & \nu_{t} &\sim \dnorm{0, v_{t}} \\
  \label{eq:16}
  \theta_t &= \theta_{t-1} + \omega_{t} & \omega_{t} &\sim \dnorm{0, w_{t}} \\
\end{align}
where $y_{t}$ and $\theta_{t}$ are scalars.
In this model, given the initial value $\theta_{0}$, the values of $\theta$ are determined by the distribution of the $\omega$, which is a distribution over the changes in $\theta$,
\begin{equation}
  \label{eq:12}
  \Delta \theta_{t} = \theta_{t} - \theta_{t - 1} = \omega_{t} \sim \dnorm{0, w_{t}} \text{.}
\end{equation}
Thus, the choice of the distributions of the $\omega$ parameters determines how the parameter $\theta$ can evolve over time.%
\footnote{Although I only consider Bayesian framework, the choice of prior distribution is equivalent to a choice of a penalty term in maximum likelihood framework \parencite{PolsonScott2010}.}

A common approach is to assume that the innovations have a constant variance over time, $w_{t} = w$  for all $t \in 1:T$.
If $\omega_{t}$ are distributed i.i.d. normal, then large changes in $\theta$ are penalized and the evolution of $\theta$ over time is smoothed.
\footnote{This is equivalent to an L2 penalty.}
The problem with assuming $\omega_{t} \sim \dnorm{0, w}$ is that it does not incorporate handle sparsity and large values well, two features which are expected in many data generating processes encounted in political science.

\subsection{Discrete Scale Mixtures}
\label{sec:discr-mixt-distr}

The first approach are selection approaches, in which the estimation technique selects which $\omega_{t}$ are non-zero (usually a small number), and then estimates the values of the non-zero innovations.
Models within this approach are usually formulated and estimated as discrete state-space Hidden Markov Models, \parencites{Chib1998}{spirling2007bayesian}{Park2011}{Park2010}{Blackwell2012}.
A downside of these methods is that they often require the number of discrete states to be set \parencite{Chib1998}, and although methods exist that relax that assumption, they are often not straighforward.

The intutition behind structural break models is that per period changes in the parameters are divided into two groups: structural breaks in which the change is non-zero, and non-structural breaks in which there is no change.
Thus structural breaks can be incorporated within the DLM model by replacing the normal distribution of $\omega_{t}$ with a mixture of normal distributions.
\begin{equation}
  \label{eq:1}
  \omega_{t} \sim p \eta_{t}  + (1 - p) \delta_{0} \\
  \eta_{t} & \sim \dnorm{0, \tau} 
\end{equation}
where $p$ is the prior probability that $\omega_{t}$ is a structural break ($\omega_{t} \neq 0$).
Equation \eqref{eq:1} can also be rewritten so that $\omega_{t}$ is conditionally gaussian,
\begin{equation}
  \label{eq:7}
  \begin{aligned}[t]
    \omega_{t} & \sim N(0, \lambda_{t} \tau) \\
    \lambda_{t} = & 
    \begin{cases}
      1 & \text{with probability $p$} \\
      0 & \text{with probability $1 - p$}
    \end{cases}
  \end{aligned}
\end{equation}
with the convention that $N(0, 0) = \delta_{0}$. 

The number of breaks is not fixed \textit{ex ante} but is itself random variable.
The value of $p$ determines the prior sparsity of changes and thus the expected number of structural breaks,
and can either be fixed by the researcher or given a prior distribution and estimated.
Prior regime duration distributed geometric with mean $(1 - p) / p$ and variance $\sqrt{1 - p} / p$ \parencite[68]{GiordaniKohn2008}, the same as in the original \textcite{Chib1998} HMM model.
If $p = 1$, then then the model is equivalent to a dynamic linear model with constant innovation variance.

\subsection{Continuous Scale Mixtures}
\label{sec:shrinkage}

The approach which will be taken in this paper, is to adopt continuous scale mixture of 
\begin{equation}
  \label{eq:4}
  \begin{aligned}[t]
    \omega_{t} &\sim \dnorm{0, \lambda_{t}^{2} \tau^{2}} \\
    \lamda_{t} &\sim p(\lambda_{t})
  \end{aligned}
\end{equation}

Since many computationally efficient forms of maximization and sampling of the dynamic linear model require the errors and innovations be distributed normal, I will focus on a class of shrinkage distributions that are scale mixtures of normal distributions, i.e. each $\omega_{t}$ will be distributed normal, but the variances of these normal distributions are drawn from a hierarchical distribution.
\begin{equation}
  \label{eq:6}
  \begin{aligned}[t]
    \omega_{t} | \tau^{2}, \lambda^{2} & \sim \dnorm{0, \sigma^{2} \lambda^{2}} \\
    \lambda_{t}^{2} & \sim p(\lambda^{2}_{t})
  \end{aligned}
\end{equation}
where $\tau^{2}$ is called the global shrinkage parameter, and $\lambda_{t}^{2}$ are called the local shrinkage parameters.
The $t$-distribution is an example of a scale mixture of normal distributions, and has been suggested for dynamic linear model estimation that is robust to structural breaks \parencites{HarveyKoopman2000}{PetrisPetroneEtAl2009}.
The $t$-distribution in its most extreme form (Cauchy), has very flat tails, which allows for structural breaks.
However, it does not have a large spike at zero, and thus may not shrink noise enough.

However, as noted before, the problem of estimating $\omega$ is an example of a large-p problem and there are many proposed distributions for shrinkage parameters.
The class of scale-normal mixtures includes many Bayesian shrinkage priors, such as the student-\textit{t} \parencite{Tipping2001}, double-exponential prior (Bayesian LASSO) \parencites{LiGoel2006}{ParkCasella2008}{Hans2009}, normal-Jeffreys \parencites{FigueiredoMember2003}{BaeMallick2004}, Strawderman-Berger \parencites{Strawderman1971}{Berger1980}, double Pareto \parencite{ArmaganDunsonLee2011},  and normal-exponential-gamma \parencite{BrownGriffin2005}, normal/gamma and normal/inverse-gamma \parencites{CaronDoucet2008}{BrownGriffin2010}.

The distribution of $\omega$ which will be used in this paper is the horseshoe prior distribution, introduced in \textcites{CarvalhoPolsonScott2009}{CarvalhoPolsonScott2010}.
The horseshoe prior distribution does not have an analytical form, but is formed when the $\lambda_{t}$ in equation \eqref{eq:6} are independently distributed half-Cauchy,
\begin{align}
  \label{eq:13}
  \lambda_{t} &\sim \dhalfcauchy{0, 1}
\end{align}
where $\dhalfcauchy{0, \gamma}$ is the standard half-Cauchy distribution with support on the positive real numbers, and scale $\gamma$.%
\footnote{
  This implies that $p(\lambda^{2})$ is distributed inverse-beta, $IB(a, b)$ where $a = b = \frac{1}{2}$ \parencite[4]{PolsonScott2010}. 
}

The horseshoe prior distribution has two features that make it useful as a shrinkage prior for sparse parameters.
It has flat Cauchy-like tails, which mean that structural breaks are not shrunk \textit{a posteriori}.
It also has an infinitely tall spike at zero which aggressively shrinks non-structural breaks to zero.
Figure \ref{fig:horseshoe} compares the density of the horseshoe prior distribution against the normal, Cauchy, and Laplacian (Baysian LASSO) distributions.
The Laplacian distribution is commonly used in sparse regularized regression.
However, the horseshoe prior distribution has both a taller spike at zero and fatter tails than the Laplacian distribution.

\begin{figure}
  \centering
  \includegraphics{plots/fig-horseshoe1.pdf}
  \includegraphics{plots/fig-horseshoe2.pdf}
  \caption{The density of the horseshoe prior distribution (in black) compared with the densities of the normal, Cauchy, and Laplacian distributions (in gray).}
  \label{fig:horseshoe}
\end{figure}

To summarize, a local level model 
\begin{equation}
  \label{eq:3}
  \begin{aligned}[t]
    y_{t} & \sim \dnorm{\theta_{t}, \sigma^{2}} \\
    \theta_{t} & \sim \dnorm{\theta_{t - 1}, \sigma^{2} \tau^{2} \lambda^{2}} \\
    \lambda_{t} & \sim \dhalfcauchy{0, 1}
  \end{aligned}
\end{equation}
The values $\sigma$ and $\tau$ can be estimated as parameters. 
In which case, the following prior distributions are suggested in \textcite{CarvalhoPolsonScott2010}{DattaGhosh2012},
\begin{align}
  \label{eq:9}
  \begin{aligned}[t]
    p(\sigma^{2}) & \frac{1}{\sigma^{2}}  \\
    \tau &\sim \dhalfcauchy{0, 1} \text{.}
  \end{aligned}
\end{align}

\subsection{Identifying Structural Breaks}
\label{sec:ident-struct-breaks}

The posterior distribution $p(\theta | y)$ can be used to identify structural breaks in two ways. 

The first method is simply to use the posterior distribution of $\omega_{t} = \theta_{t} - \theta_{t-1}$.
The probability that there was a positive change in $\theta$ at time $t$ is
$\Pr(\omega_{t} | y) > 0$, and the probability of a negative change is $\Pr(\omega_{t} | y) < 0$.
Alternatively, the highest posterior density region of the posterior can be calculated and the researcher can classify observations as structural breaks if the credible interval for a given probability does not cross zero.

The second method is to calculate a quantity similar to the posterior probability $\omega \neq 0$ that would be produced by a spike and slab distribution.
For the discrete mixture model in equation \eqref{eq:1}, if $g$ is sufficiently heavy-tailed, the posterior mean,
\begin{equation}
  \label{eq:17}
  \E(\omega_{t} | y) \approx p_{t} e_{t} 
\end{equation}
where $p_{t}$ is the posterior probability of $\omega_{t} \neq 0$.
For a scale mixture of normal distributions, the expected value of $\omega_{t}$ is 
\begin{equation}
  \label{eq:10}
  E(\omega_{t} | y) =
  \left(
    1 - \frac{1}{1 + \lambda^{2}_{t} \tau^{2}}
  \right) e_{t} = (1 - \hat \kappa_{t}) e_{i}
\end{equation}
Equating equation \eqref{eq:17} with equation \eqref{eq:10}, it is clear that 
the quantity $1 - \hat \kappa_{t}$ behaves similarly to $p_{t}$.
While $1 - \hat \kappa_{t}$ can be calculated for all mixture of normal distributions, it is not necessarily the case that $1 - \hat \kappa_{t} \approx p_{t}$.
However, one of the advantages of using the Horseshoe Prior distribution is that 
\parencite[474]{CarvalhoPolsonScott2010} show through simulations that this is the case for the Horseshoe Prior distribution.
Given that, \textcite{CarvalhoPolsonScott2010} recommend the following  decision rule under a 0-1 loss function to classify whether an observation is a signal (structural break in this case) or noise (no structural break), 
\begin{equation}
  \label{eq:5}
  \text{$H_{0,t}$ if $\nu_{t} = 1 - E(\kappa_{t}|y_{t}, \nu_{t-1} \lambda_{t}, \tau, \sigma) > \frac{1}{2}$}
\end{equation}

\section{Monte Carlo}
\label{sec:monte-carlo}

This paper probably needs a Monte Carlo.

\section{Examples}
\label{sec:examples}

\subsection{Nile Flow Data}
\label{sec:nile}

The first example is the Nile river flow data, which is a classic dataset in the  \parencites{Cobb1978}{Balke1993}{JongPenzer1998}{DurbinKoopman2001}{DurbinKoopman2012}
The data consist of annual observations of the flow of the Nile river at Ashwan between 1871 and 1970.
It is well known that there was a level shift in 1899, both due to the construction of a damn at Ashwan and weather changes.
Since in this example, there is only one clear change point, it is not fully exploiting the flexibility of this method, but is instead a sanity check.
It will illustrate important differences between the adaption of the HPDLM and GDLM to a structural break,
and it will show how, the HPDLM can approximate an intervention without any \textit{ex ante} input from the analyst.

I compare the performance of the horseshoe prior innovations model ($M_{nile,HS}$) with two alternative models.
The first model has a uses a normal distribution with a time-invariant variance for the innovations ($M_{nile,normal}$).
The second model extends $M_{nile,normal}$ to include a single parameter that represents change in the level after 1899 ($M_{nile,normal2}$).
The details of these models is given in Section \ref{sec:nile-1}.
Figure \ref{fig:nile} plots the original data, and the mean of the posterior predictive distributions for each of these models.
The  shows a sharp drop 1899 and stability before and after the break.
The normal model $M_{nile,normal}$ shows a smother adjustment with the decline in the level beginning a few years before 1899 and continuing a few years thereafter.
Model $M_{nile,normal}$ also shows more variability in the level before and after 1899.
This variability illustrates the importance of using a scale mixtures of normal distributions with local shrinkage parameters ($\lambda_{t}$) in addition to a global shrinkage parameter ($\tau$).
Since the normal model has only a single global shrinkage parameter ($\tau$). 
In order to accommodate the large change in the level in 1899, the estimated value of $\tau$ must increase.
However, increasing $\tau$ will result in less smoothing in the other observations.
The normal model must trade off shrinking the non-structural breaks and not shrinking the structural break with only a single parameter, resulting in over-smoothing around the break and under-smoothing elsewhere.

It is also remarkable that the horseshoe prior model's posterior predictive means closely match those of the intervention model ($M_{nile,normal2}$) without any \textit{ex ante} knowledge of the presence of the structural break in 1899.
There is a slight difference in the two models in that the horseshoe prior model puts some weight on the possibility that the structural break occurred in 1897 or 1898.
This is most likely due to the low signal to noise ratio in the data; note that the observations in 1897 and 1898 are consistent with, although high for, the distribution of flows after 1899.
The Nile model is an easy case in that the series has a single, large level change with a clear causal event, and thus easy to include a dummy variable.
However, in many applications, the presence of the structural break will not be known, and in fact estimating the presence and location of the structural breaks will be the purpose of the application.

Figures \ref{fig:nile_innovations} and \ref{fig:nile_w} show the results of the two methods that could be used to identify structural breaks. 
Figure \ref{fig:nile_innovations} plots the mean and 95 percent HPD interval of each $p(\omega_{t} | y)$.%
\footnote{For $M_{nile,normal2}$, the posterior distribution $p(\omega_{t} + \delta (x_{t} - x_{t-1}) | y)$ is used.}
The normal model $M_{nile,nomral}$ shows no structural breaks, while the intervention model $M_{nile,normal2}$ shows a clear structural break at 1899.
In the horseshoe model, the estimated mean of $p(\omega_{1899} | y)$ is large, suggesting a structural break, although its 95 percent credible interval does not cross zero.
As noted before, this seems to be due to small, although highly variable estimates, of $\omega$ in 1897 and 1898.
However, the second method, using the values of $w_{t}$ classifies 1899 as a structural break, giving the $\Pr(\omega_{1899} \neq 0) \approx 0.55$.

\begin{figure}[htpb]
  \centering
  \includegraphics{plots/fig-nile.pdf}
  \caption{Plot of mean posterior predictive distributions ($\E p(\tilde{y}| y)$) for the normal, normal2, and horseshoe prior distribution models.}
  \label{fig:nile}
\end{figure}

\begin{figure}[htpb]
  \centering
  \includegraphics{plots/fig-nile_w.pdf}
  \caption{Plot of estimated probability of a structural break, calculated as $w_{i} = 1 - \E(\hat{\kappa})$}
  \label{fig:nile_w}
\end{figure}

\begin{figure}[htpb]
  \centering
  \includegraphics{plots/fig-nile_innovations.pdf}
  \caption{Plot of innovations}
  \label{fig:nile_innovations}
\end{figure}

\begin{table}[htpb]
  \centering
  % latex table generated in R 3.0.0 by xtable 1.7-1 package
% Sun May  5 02:57:42 2013
\begin{tabular}{rrrrrr}
  \hline
 & WAIC & $p$ & $L$ & MSE & $\chi^2$ \\ 
  \hline
normal & 1259 & 15 & -614 & 11999 & 68 \\ 
  normal2 & 1259 & 6 & -623 & 15060 & 90 \\ 
  hs & 1261 & 9 & -622 & 14478 & 82 \\ 
   \hline
\end{tabular}

  \caption{Model summary statistics of Nile models.}
  \label{tab:nile}
\end{table}

\clearpage{}

\subsubsection{Other Examples}

This paper needs a political science example.

\begin{itemize}
\item Presidential approval for George W. Bush. \parencites{RatkovicEng2010}
\item Median ideal point of the Senate. \parencites{RatkovicEng2010}
\item Supreme court dissents and concurrences. 1 or 2 structural breaks. Poisson data. \parencite{CalderiaZorn1998}
\item Discrete DV change-point models in \parencite{spirling2007bayesian}.
\item Interest Rates, Inflation, and GDP growth are common economics examples, e.g. \textcite{GiordaniKohn2008}.
\end{itemize}

\section{Appendix}
\label{sec:appendix}


\subsection{Models}
\label{sec:models}

\subsubsection{Nile}
\label{sec:nile-1}

Model $M_{nile,normal}$ is a local level model with a normal distribution.
The observation variance is given an improper Jeffrey's prior.
The system variance (global scale parameter) is given a half-Cauchy distribution. 
The initial state is given a semi-informative prior, a normal distribution with a mean at the value of the first observation, and variance equal to the sample variance of the data.
\begin{equation}
  \label{eq:11}
  \begin{aligned}[t]
    y_{t} &\sim \dnorm{\theta_{t}, \sigma^{2}} \\
    \theta_{t} &\sim \dnorm{\theta_{t - 1}, \sigma^{2} \tau^{2}} \\
    p(\sigma^{2}) &= \frac{1}{\sigma^{2}} \\
    \tau &\sim \dhalfcauchy{0, 1} \\
    p(\theta_{1}) &\sim \dnorm{y_{1}, \var{y}}
  \end{aligned}
\end{equation}

Model $M_{nile,HS}$ differs from $M_{nile,normal}$ in that it assumes a horseshoe prior distribution on 
the innovations,
\begin{equation}
  \label{eq:18}
  \begin{aligned}[t]
    \theta_{t} &\sim \dnorm{\theta_{t - 1}, \sigma^{2} \lambda_{t}^{2} \tau^{2}} \\
    \lambda &\sim \dhalfcauchy{0, 1} \\
    \tau &\sim \dhalfcauchy{0, 1}
  \end{aligned}
\end{equation}

Model $M_{nile,HS}$ differs from $M_{nile,normal}$ by adding an intervention parameter $\delta$ to the observation equation to model the level shift. 
The data $x_{t}$ is a binary vector equal to 0 before 1899, and 1 thereafter.
\begin{equation}
  \label{eq:18}
  \begin{aligned}[t]
    y_{t} &\sim \dnorm{\theta_{t} + \delta x_{t}, \sigma^{2}} \\
    \delta &\sim \dunif{-\infty, \infty}
  \end{aligned}
\end{equation}


\subsubsection{CP6}
\label{sec:cp6}


% \subsection{Code}
% \label{sec:code}

% \subsection{Normal DLM}

% \begin{Verbatim}[commandchars=\\\{\}]
\PY{k+kn}{data} \PY{p}{\PYZob{}}
  \PY{c+c1}{// number of observations}
  \PY{k+kt}{int} \PY{n}{n\PYZus{}obs}\PY{p}{;}
  \PY{c+c1}{// observed data}
  \PY{k+kt}{real} \PY{n}{y}\PY{p}{[}\PY{n}{n\PYZus{}obs}\PY{p}{]}\PY{p}{;}
  \PY{c+c1}{// initial values}
  \PY{k+kt}{real} \PY{n}{theta1\PYZus{}mean}\PY{p}{;}
  \PY{k+kt}{real}\PY{o}{\PYZlt{}}\PY{k}{lower}\PY{o}{=}\PY{l+m+mf}{0.0}\PY{o}{\PYZgt{}} \PY{n}{theta1\PYZus{}sd}\PY{p}{;}
\PY{p}{\PYZcb{}}
\PY{k+kn}{parameters} \PY{p}{\PYZob{}}
  \PY{c+c1}{// latent states}
  \PY{k+kt}{real} \PY{n}{theta\PYZus{}innov}\PY{p}{[}\PY{n}{n\PYZus{}obs}\PY{p}{]}\PY{p}{;}
  \PY{k+kt}{real} \PY{n}{logsigma}\PY{p}{;}
  \PY{k+kt}{real}\PY{o}{\PYZlt{}}\PY{k}{lower}\PY{o}{=}\PY{l+m+mf}{0.0}\PY{o}{\PYZgt{}} \PY{n}{tau}\PY{p}{;}
\PY{p}{\PYZcb{}}
\PY{k+kn}{transformed parameters} \PY{p}{\PYZob{}}
  \PY{k+kt}{real}\PY{o}{\PYZlt{}}\PY{k}{lower}\PY{o}{=}\PY{l+m+mf}{0.0}\PY{o}{\PYZgt{}} \PY{n}{sigma}\PY{p}{;}
  \PY{k+kt}{real} \PY{n}{theta}\PY{p}{[}\PY{n}{n\PYZus{}obs}\PY{p}{]}\PY{p}{;}
  \PY{n}{sigma} \PY{o}{\PYZlt{}\PYZhy{}} \PY{n+nb}{exp}\PY{p}{(}\PY{n}{logsigma}\PY{p}{)}\PY{p}{;}

  \PY{n}{theta}\PY{p}{[}\PY{l+m+mi}{1}\PY{p}{]} \PY{o}{\PYZlt{}\PYZhy{}} \PY{n}{theta1\PYZus{}mean} \PY{o}{+} \PY{n}{theta1\PYZus{}sd} \PY{o}{*} \PY{n}{theta\PYZus{}innov}\PY{p}{[}\PY{l+m+mi}{1}\PY{p}{]}\PY{p}{;}
  \PY{k}{for} \PY{p}{(}\PY{n}{i} \PY{k}{in} \PY{l+m+mi}{2}\PY{p}{:}\PY{n}{n\PYZus{}obs}\PY{p}{)} \PY{p}{\PYZob{}}
    \PY{n}{theta}\PY{p}{[}\PY{n}{i}\PY{p}{]} \PY{o}{\PYZlt{}\PYZhy{}} \PY{n}{theta}\PY{p}{[}\PY{n}{i}\PY{l+m+mi}{\PYZhy{}1}\PY{p}{]} \PY{o}{+} \PY{n}{sigma} \PY{o}{*} \PY{n}{tau} \PY{o}{*} \PY{n}{theta\PYZus{}innov}\PY{p}{[}\PY{n}{i}\PY{p}{]}\PY{p}{;}
  \PY{p}{\PYZcb{}}
\PY{p}{\PYZcb{}}
\PY{k+kn}{model} \PY{p}{\PYZob{}}
  \PY{n}{theta\PYZus{}innov} \PY{o}{\PYZti{}} \PY{n+nb}{normal}\PY{p}{(}\PY{l+m+mf}{0.0}\PY{p}{,} \PY{l+m+mf}{1.0}\PY{p}{)}\PY{p}{;}
  \PY{n}{tau} \PY{o}{\PYZti{}} \PY{n+nb}{cauchy}\PY{p}{(}\PY{l+m+mf}{0.0}\PY{p}{,} \PY{l+m+mf}{1.0}\PY{p}{)}\PY{p}{;}
  \PY{n}{y} \PY{o}{\PYZti{}} \PY{n+nb}{normal}\PY{p}{(}\PY{n}{theta}\PY{p}{,} \PY{n}{sigma}\PY{p}{)}\PY{p}{;}
\PY{p}{\PYZcb{}}
\PY{k+kn}{generated quantities} \PY{p}{\PYZob{}}
  \PY{k+kt}{real} \PY{n}{llik}\PY{p}{[}\PY{n}{n\PYZus{}obs}\PY{p}{]}\PY{p}{;}
  \PY{k+kt}{real} \PY{n}{deviance}\PY{p}{;}
  
  \PY{k}{for} \PY{p}{(}\PY{n}{i} \PY{k}{in} \PY{l+m+mi}{1}\PY{p}{:}\PY{n}{n\PYZus{}obs}\PY{p}{)} \PY{p}{\PYZob{}}
    \PY{n}{llik}\PY{p}{[}\PY{n}{i}\PY{p}{]} \PY{o}{\PYZlt{}\PYZhy{}} \PY{n+nb}{normal\PYZus{}log}\PY{p}{(}\PY{n}{y}\PY{p}{[}\PY{n}{i}\PY{p}{]}\PY{p}{,} \PY{n}{theta}\PY{p}{[}\PY{n}{i}\PY{p}{]}\PY{p}{,} \PY{n}{sigma}\PY{p}{)}\PY{p}{;}
  \PY{p}{\PYZcb{}}
  \PY{n}{deviance} \PY{o}{\PYZlt{}\PYZhy{}} \PY{l+m+mi}{\PYZhy{}2} \PY{o}{*} \PY{n+nb}{sum}\PY{p}{(}\PY{n}{llik}\PY{p}{)}\PY{p}{;}
\PY{p}{\PYZcb{}}
\end{Verbatim}



% \subsection{Horseshoe Prior DLM}

% \begin{Verbatim}[commandchars=\\\{\}]
\PY{k+kn}{data} \PY{p}{\PYZob{}}
  \PY{c+c1}{// number of observations}
  \PY{k+kt}{int} \PY{n}{n\PYZus{}obs}\PY{p}{;}
  \PY{c+c1}{// observed data}
  \PY{k+kt}{real} \PY{n}{y}\PY{p}{[}\PY{n}{n\PYZus{}obs}\PY{p}{]}\PY{p}{;}
  \PY{c+c1}{// initial values}
  \PY{k+kt}{real} \PY{n}{theta1\PYZus{}mean}\PY{p}{;}
  \PY{k+kt}{real}\PY{o}{\PYZlt{}}\PY{k}{lower}\PY{o}{=}\PY{l+m+mf}{0.0}\PY{o}{\PYZgt{}} \PY{n}{theta1\PYZus{}sd}\PY{p}{;}
\PY{p}{\PYZcb{}}
\PY{k+kn}{parameters} \PY{p}{\PYZob{}}
  \PY{c+c1}{// latent states}
  \PY{k+kt}{real} \PY{n}{theta\PYZus{}innov}\PY{p}{[}\PY{n}{n\PYZus{}obs}\PY{p}{]}\PY{p}{;}
  \PY{c+c1}{// measurement error}
  \PY{k+kt}{real} \PY{n}{logsigma}\PY{p}{;}
  \PY{c+c1}{// system error}
  \PY{k+kt}{real}\PY{o}{\PYZlt{}}\PY{k}{lower}\PY{o}{=}\PY{l+m+mf}{0.0}\PY{o}{\PYZgt{}} \PY{n}{lambda}\PY{p}{[}\PY{n}{n\PYZus{}obs} \PY{l+m+mi}{\PYZhy{}1}\PY{p}{]}\PY{p}{;}
  \PY{k+kt}{real}\PY{o}{\PYZlt{}}\PY{k}{lower}\PY{o}{=}\PY{l+m+mf}{0.0}\PY{o}{\PYZgt{}} \PY{n}{tau}\PY{p}{;}
\PY{p}{\PYZcb{}}
\PY{k+kn}{transformed parameters} \PY{p}{\PYZob{}}
  \PY{k+kt}{real}\PY{o}{\PYZlt{}}\PY{k}{lower}\PY{o}{=}\PY{l+m+mf}{0.0}\PY{o}{\PYZgt{}} \PY{n}{sigma}\PY{p}{;}
  \PY{k+kt}{real} \PY{n}{theta}\PY{p}{[}\PY{n}{n\PYZus{}obs}\PY{p}{]}\PY{p}{;}

  \PY{n}{sigma} \PY{o}{\PYZlt{}\PYZhy{}} \PY{n+nb}{exp}\PY{p}{(}\PY{n}{logsigma}\PY{p}{)}\PY{p}{;}

  \PY{n}{theta}\PY{p}{[}\PY{l+m+mi}{1}\PY{p}{]} \PY{o}{\PYZlt{}\PYZhy{}} \PY{n}{theta1\PYZus{}mean} \PY{o}{+} \PY{n}{theta1\PYZus{}sd} \PY{o}{*} \PY{n}{theta\PYZus{}innov}\PY{p}{[}\PY{l+m+mi}{1}\PY{p}{]}\PY{p}{;}
  \PY{k}{for} \PY{p}{(}\PY{n}{i} \PY{k}{in} \PY{l+m+mi}{2}\PY{p}{:}\PY{n}{n\PYZus{}obs}\PY{p}{)} \PY{p}{\PYZob{}}
    \PY{n}{theta}\PY{p}{[}\PY{n}{i}\PY{p}{]} \PY{o}{\PYZlt{}\PYZhy{}} \PY{n}{theta}\PY{p}{[}\PY{n}{i}\PY{l+m+mi}{\PYZhy{}1}\PY{p}{]} \PY{o}{+} \PY{n}{lambda}\PY{p}{[}\PY{n}{i}\PY{l+m+mi}{\PYZhy{}1}\PY{p}{]} \PY{o}{*} \PY{n}{tau} \PY{o}{*} \PY{n}{sigma} \PY{o}{*} \PY{n}{theta\PYZus{}innov}\PY{p}{[}\PY{n}{i}\PY{p}{]}\PY{p}{;}
  \PY{p}{\PYZcb{}}
\PY{p}{\PYZcb{}}
\PY{k+kn}{model} \PY{p}{\PYZob{}}
  \PY{n}{theta\PYZus{}innov} \PY{o}{\PYZti{}} \PY{n+nb}{normal}\PY{p}{(}\PY{l+m+mf}{0.0}\PY{p}{,} \PY{l+m+mf}{1.0}\PY{p}{)}\PY{p}{;}
  \PY{c+c1}{// half\PYZhy{}cauchy since lambda \PYZgt{} 0.}
  \PY{n}{lambda} \PY{o}{\PYZti{}} \PY{n+nb}{cauchy}\PY{p}{(}\PY{l+m+mf}{0.0}\PY{p}{,} \PY{l+m+mf}{1.0}\PY{p}{)}\PY{p}{;}
  \PY{n}{tau} \PY{o}{\PYZti{}} \PY{n+nb}{cauchy}\PY{p}{(}\PY{l+m+mf}{0.0}\PY{p}{,} \PY{l+m+mf}{1.0}\PY{p}{)}\PY{p}{;}
  \PY{n}{y} \PY{o}{\PYZti{}} \PY{n+nb}{normal}\PY{p}{(}\PY{n}{theta}\PY{p}{,} \PY{n}{sigma}\PY{p}{)}\PY{p}{;}
\PY{p}{\PYZcb{}}
\PY{k+kn}{generated quantities} \PY{p}{\PYZob{}}
  \PY{k+kt}{real} \PY{n}{llik}\PY{p}{[}\PY{n}{n\PYZus{}obs}\PY{p}{]}\PY{p}{;}
  \PY{k+kt}{real} \PY{n}{deviance}\PY{p}{;}
  \PY{k+kt}{real} \PY{n}{kappa}\PY{p}{[}\PY{n}{n\PYZus{}obs}\PY{p}{]}\PY{p}{;}

  \PY{k}{for} \PY{p}{(}\PY{n}{i} \PY{k}{in} \PY{l+m+mi}{1}\PY{p}{:}\PY{n}{n\PYZus{}obs}\PY{p}{)} \PY{p}{\PYZob{}}
    \PY{n}{llik}\PY{p}{[}\PY{n}{i}\PY{p}{]} \PY{o}{\PYZlt{}\PYZhy{}} \PY{n+nb}{normal\PYZus{}log}\PY{p}{(}\PY{n}{y}\PY{p}{[}\PY{n}{i}\PY{p}{]}\PY{p}{,} \PY{n}{theta}\PY{p}{[}\PY{n}{i}\PY{p}{]}\PY{p}{,} \PY{n}{sigma}\PY{p}{)}\PY{p}{;}
  \PY{p}{\PYZcb{}}
  \PY{n}{deviance} \PY{o}{\PYZlt{}\PYZhy{}} \PY{l+m+mi}{\PYZhy{}2} \PY{o}{*} \PY{n+nb}{sum}\PY{p}{(}\PY{n}{llik}\PY{p}{)}\PY{p}{;}
  \PY{p}{\PYZob{}}
    \PY{k+kt}{real} \PY{n}{sigma2}\PY{p}{;}
    \PY{n}{sigma2} \PY{o}{\PYZlt{}\PYZhy{}} \PY{n+nb}{pow}\PY{p}{(}\PY{n}{sigma}\PY{p}{,} \PY{l+m+mf}{2.0}\PY{p}{)}\PY{p}{;}
    \PY{n}{kappa}\PY{p}{[}\PY{l+m+mi}{1}\PY{p}{]} \PY{o}{\PYZlt{}\PYZhy{}} \PY{n}{sigma2} \PY{o}{/} \PY{p}{(}\PY{n}{sigma2} \PY{o}{+} \PY{n+nb}{pow}\PY{p}{(}\PY{n}{theta1\PYZus{}sd}\PY{p}{,} \PY{l+m+mi}{2}\PY{p}{)}\PY{p}{)}\PY{p}{;}
    \PY{k}{for} \PY{p}{(}\PY{n}{i} \PY{k}{in} \PY{l+m+mi}{2}\PY{p}{:}\PY{n}{n\PYZus{}obs}\PY{p}{)} \PY{p}{\PYZob{}}
      \PY{n}{kappa}\PY{p}{[}\PY{n}{i}\PY{p}{]} \PY{o}{\PYZlt{}\PYZhy{}} \PY{l+m+mi}{1} \PY{o}{/} \PY{p}{(}\PY{l+m+mi}{1} \PY{o}{+} \PY{n+nb}{pow}\PY{p}{(}\PY{n}{lambda}\PY{p}{[}\PY{n}{i} \PY{o}{\PYZhy{}} \PY{l+m+mi}{1}\PY{p}{]}\PY{p}{,} \PY{l+m+mi}{2}\PY{p}{)} \PY{o}{*} \PY{n+nb}{pow}\PY{p}{(}\PY{n}{tau}\PY{p}{,} \PY{l+m+mi}{2}\PY{p}{)}\PY{p}{)}\PY{p}{;}
    \PY{p}{\PYZcb{}}
  \PY{p}{\PYZcb{}}
\PY{p}{\PYZcb{}}
\end{Verbatim}


\printbibliography{}

\end{document}

%%% Local Variables: 
%%% mode: latex
%%% TeX-master: t
%%% End: 

%  LocalWords:  Carvallho
